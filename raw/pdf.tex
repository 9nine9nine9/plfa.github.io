\documentclass[10pt]{book}

\usepackage{hyperref}
\usepackage[links]{agda}
\usepackage{fontspec}
\setmainfont{DejaVu Sans}
\setsansfont{DejaVu Sans}
\setmonofont{[DejaVu-mononoki-Symbola-Droid.ttf]}
\usepackage{soul}
\usepackage{tcolorbox}
\tcbuselibrary{skins,breakable}
\usepackage{fancyvrb}
\usepackage{xcolor}
\usepackage{tikz}
\usepackage{setspace}
\usepackage{geometry}
\geometry{
  a4paper,
  total={170mm,257mm},
  left=20mm,
  top=20mm,
}


% Wrap texttt lines
\sloppy

% Disable section numbering
\setcounter{secnumdepth}{0}

% Set the global text color:
\definecolor{textcolor}{HTML}{111111}
\color{textcolor}

% Change background color for inline code in markdown files.
% The following code does not work well for long text as the
% text will exceed the page boundary.
\definecolor{background-color}{HTML}{EEEEFF}
\let\oldtexttt\texttt%
\renewcommand{\texttt}[1]{\colorbox{background-color}{\oldtexttt{#1}}}

% Box with background colour similar to web version:
\newtcolorbox{agda}[1][]{
  frame hidden,
  colback=background-color,
  spartan,
  left=5pt,
  boxrule=0pt,
  breakable,
}

% Verbatim environment similarly indented to Agda code blocks.
\DefineVerbatimEnvironment{verbatim}{Verbatim}{xleftmargin=0pt}%

% Adding backgrounds to verbatim environments.
\newenvironment{pre}{
  \VerbatimEnvironment
  \begin{agda}
  \begin{verbatim}
}{\end{verbatim}
  \end{agda}
}

% Use special font families without TeX ligatures for Agda code.
% Solution inspired by a comment by Enrico Gregorio:
% https://tex.stackexchange.com/a/103078
\newfontfamily{\AgdaSerifFont}{DejaVu-Serif}
\newfontfamily{\AgdaSansSerifFont}{DejaVu-Sans}
\newfontfamily{\AgdaTypewriterFont}{DejaVu-mononoki-Symbola-Droid}
\renewcommand{\AgdaFontStyle}[1]{{\AgdaTypewriterFont{}#1}}
\renewcommand{\AgdaKeywordFontStyle}[1]{{\AgdaTypewriterFont{}#1}}
\renewcommand{\AgdaStringFontStyle}[1]{{\AgdaTypewriterFont{}#1}}
\renewcommand{\AgdaCommentFontStyle}[1]{{\AgdaTypewriterFont{}#1}}
\renewcommand{\AgdaBoundFontStyle}[1]{{\AgdaTypewriterFont{}#1}}

\AgdaNoSpaceAroundCode{}

% Adjust spacing after toc numbering
\usepackage{tocloft}
\setlength\cftchapnumwidth{3em}
\cftsetpnumwidth{4em}

% Style links with colors instead of boxes:
% https://tex.stackexchange.com/q/823
\definecolor{linkcolor}{HTML}{2A7AE2}
\hypersetup{
  colorlinks,
  linkcolor={linkcolor},
  citecolor={linkcolor},
  urlcolor={linkcolor}
}

\begin{document}

% Adjust indentation of Agda code blocks
\setlength{\mathindent}{0pt}
\setlength{\parindent}{0em}
\setlength{\parskip}{1em}

% Provide \tightlist environment
% https://tex.stackexchange.com/q/257418
\providecommand{\tightlist}{%
  \setlength{\itemsep}{0pt}\setlength{\parskip}{0pt}}

% Based on \titleAM:
% https://ctan.org/pkg/titlepages
\begin{titlepage}
  \newlength{\drop}%
  \setlength{\drop}{0.12\textheight}%
  \centering%
  \vspace*{\drop}
  \begingroup% Ancient Mariner, AW p. 151
  {\large Philip Wadler, Wen Kokke, and Jeremy G. Siek}\\[\baselineskip]
  {\Huge PROGRAMMING LANGUAGE}\\[\baselineskip]
  {\Huge FOUNDATIONS}\\[\baselineskip]
  {\Large IN}\\[\baselineskip]
  {\Huge Agda}\\[\drop]
  \vfill%
  {\small\scshape 2021}\par%
  \null\endgroup
\end{titlepage}



% Open front, main, and back matter sections:

\frontmatter%
\setcounter{tocdepth}{0}

\tableofcontents%
\setcounter{tocdepth}{1}




% Only print title for main and back matter:

% NOTE: Front matter titles are inserted via book/lua/single-file-headers.lua.


% Include each section.

\hypertarget{dedication}{%
  \chapter{Dedication}\label{dedication}}
\input{plfa/frontmatter/dedication.tex}

\hypertarget{preface}{%
  \chapter{Preface}\label{preface}}
\input{plfa/frontmatter/preface.tex}

\hypertarget{gettingstarted}{%
  \chapter{Getting Started}\label{gettingstarted}}
\input{plfa/frontmatter/README.tex}


% Close front, main, and back matter sections:






% Open front, main, and back matter sections:


\mainmatter%



% Only print title for main and back matter:

\part{Part 1: Logical Foundations}


% Include each section.

\hypertarget{naturals}{%
  \chapter{Naturals: Natural numbers}\label{naturals}}
\input{plfa/part1/Naturals.tex}

\hypertarget{induction}{%
  \chapter{Induction: Proof by Induction}\label{induction}}
\input{plfa/part1/Induction.tex}

\hypertarget{relations}{%
  \chapter{Relations: Inductive definition of relations}\label{relations}}
\input{plfa/part1/Relations.tex}

\hypertarget{equality}{%
  \chapter{Equality: Equality and equational reasoning}\label{equality}}
\input{plfa/part1/Equality.tex}

\hypertarget{isomorphism}{%
  \chapter{Isomorphism: Isomorphism and Embedding}\label{isomorphism}}
\input{plfa/part1/Isomorphism.tex}

\hypertarget{connectives}{%
  \chapter{Connectives: Conjunction, disjunction, and implication}\label{connectives}}
\input{plfa/part1/Connectives.tex}

\hypertarget{negation}{%
  \chapter{Negation: Negation, with intuitionistic and classical logic}\label{negation}}
\input{plfa/part1/Negation.tex}

\hypertarget{quantifiers}{%
  \chapter{Quantifiers: Universals and existentials}\label{quantifiers}}
\input{plfa/part1/Quantifiers.tex}

\hypertarget{decidable}{%
  \chapter{Decidable: Booleans and decision procedures}\label{decidable}}
\input{plfa/part1/Decidable.tex}

\hypertarget{lists}{%
  \chapter{Lists: Lists and higher-order functions}\label{lists}}
\input{plfa/part1/Lists.tex}


% Close front, main, and back matter sections:


\cleardoublepage%
\phantomsection%




% Open front, main, and back matter sections:




% Only print title for main and back matter:

\part{Part 2: Programming Language Foundations}


% Include each section.

\hypertarget{lambda}{%
  \chapter{Lambda: Introduction to Lambda Calculus}\label{lambda}}
\input{plfa/part2/Lambda.tex}

\hypertarget{properties}{%
  \chapter{Properties: Progress and Preservation}\label{properties}}
\input{plfa/part2/Properties.tex}

\hypertarget{debruijn}{%
  \chapter{DeBruijn: Intrinsically-typed de Bruijn representation}\label{debruijn}}
\input{plfa/part2/DeBruijn.tex}

\hypertarget{more}{%
  \chapter{More: Additional constructs of simply-typed lambda calculus}\label{more}}
\input{plfa/part2/More.tex}

\hypertarget{bisimulation}{%
  \chapter{Bisimulation: Relating reduction systems}\label{bisimulation}}
\input{plfa/part2/Bisimulation.tex}

\hypertarget{inference}{%
  \chapter{Inference: Bidirectional type inference}\label{inference}}
\input{plfa/part2/Inference.tex}

\hypertarget{untyped}{%
  \chapter{Untyped: Untyped lambda calculus with full normalisation}\label{untyped}}
\input{plfa/part2/Untyped.tex}

\hypertarget{confluence}{%
  \chapter{Confluence: Confluence of untyped lambda calculus}\label{confluence}}
\input{plfa/part2/Confluence.tex}

\hypertarget{bigstep}{%
  \chapter{BigStep: Big-step semantics of untyped lambda calculus}\label{bigstep}}
\input{plfa/part2/BigStep.tex}


% Close front, main, and back matter sections:





% Open front, main, and back matter sections:




% Only print title for main and back matter:

\part{Part 3: Denotational Semantics}


% Include each section.

\hypertarget{denotational}{%
  \chapter{Denotational: Denotational semantics of untyped lambda calculus}\label{denotational}}
\input{plfa/part3/Denotational.tex}

\hypertarget{compositional}{%
  \chapter{Compositional: The denotational semantics is compositional}\label{compositional}}
\input{plfa/part3/Compositional.tex}

\hypertarget{soundness}{%
  \chapter{Soundness: Soundness of reduction with respect to denotational semantics}\label{soundness}}
\input{plfa/part3/Soundness.tex}

\hypertarget{adequacy}{%
  \chapter{Adequacy: Adequacy of denotational semantics with respect to operational semantics}\label{adequacy}}
\input{plfa/part3/Adequacy.tex}

\hypertarget{contextualequivalence}{%
  \chapter{ContextualEquivalence: Denotational equality implies contextual equivalence}\label{contextualequivalence}}
\input{plfa/part3/ContextualEquivalence.tex}


% Close front, main, and back matter sections:





% Open front, main, and back matter sections:



\appendix
\addcontentsline{toc}{part}{Appendices}


% Only print title for main and back matter:

\part{Appendix}


% Include each section.

\hypertarget{substitution}{%
  \chapter{Substitution: Substitution in the untyped lambda calculus}\label{substitution}}
\input{plfa/part2/Substitution.tex}


% Close front, main, and back matter sections:






% Open front, main, and back matter sections:




% Only print title for main and back matter:

\part{Back matter}


% Include each section.

\hypertarget{acknowledgements}{%
  \chapter{Acknowledgements}\label{acknowledgements}}
\input{plfa/backmatter/acknowledgements.tex}

\hypertarget{fonts}{%
  \chapter{Fonts}\label{fonts}}
\input{plfa/backmatter/Fonts.tex}


% Close front, main, and back matter sections:





\end{document}
