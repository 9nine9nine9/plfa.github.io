\documentclass[10pt]{book}

\usepackage{hyperref}
\usepackage[links]{agda}

% Use DejaVu and Frankenfont
\usepackage{fontspec}
\setmainfont[
  Extension      = .ttf,
  Path           = fonts/,
  UprightFont    = *-Regular,
  BoldFont       = *-Bold,
  ItalicFont     = *-Italic,
  BoldItalicFont = *-BoldItalic
  ]{DejaVuSerif}
\setsansfont[
  Extension      = .ttf,
  Path           = fonts/,
  UprightFont    = *-Regular,
  BoldFont       = *-Bold,
  ItalicFont     = *-Oblique,
  BoldItalicFont = *-BoldOblique
  ]{DejaVuSans}
\setmonofont[
  Extension      = .ttf,
  UprightFont    = *-Regular,
  Path           = fonts/
  ]{Frankenfont}

% Use soul for underlines
\usepackage{soul}

% Setup colour boxes for code blocks
\usepackage{xcolor}
\usepackage{tcolorbox}
\tcbuselibrary{skins,breakable}
\usepackage{fancyvrb}
\usepackage{tikz}
\usepackage{setspace}
\usepackage[
  a4paper,
  total = {170mm,257mm},
  left  = 20mm,
  top   = 20mm,
  ]{geometry}

% Wrap texttt lines
\sloppy

% Disable section numbering
\setcounter{secnumdepth}{0}

% Set the global text color
\definecolor{TEXTCOLOR}{HTML}{111111}
\color{TEXTCOLOR}

% Change background color for inline code in markdown files.
% The following code does not work well for long text as the
% text will exceed the page boundary.
\definecolor{BACKGROUND-COLOR}{HTML}{EEEEFF}
\let\oldtexttt\texttt%
\renewcommand{\texttt}[1]{\colorbox{BACKGROUND-COLOR}{\oldtexttt{#1}}}

% Box with background colour similar to web version:
\newtcolorbox{agda}[1][]{
  frame hidden,
  colback=BACKGROUND-COLOR,
  spartan,
  left=5pt,
  boxrule=0pt,
  breakable,
}

% Verbatim environment similarly indented to Agda code blocks.
\DefineVerbatimEnvironment{verbatim}{Verbatim}{xleftmargin=0pt}%

% Adding backgrounds to verbatim environments.
\newenvironment{pre}{
  \VerbatimEnvironment
  \begin{agda}
  \begin{verbatim}
}{\end{verbatim}
  \end{agda}
}

% Use special font families without TeX ligatures for Agda code.
% Solution inspired by a comment by Enrico Gregorio:
% https://tex.stackexchange.com/a/103078
\newfontfamily{\AgdaSerifFont}{DejaVu-Serif}
\newfontfamily{\AgdaSansSerifFont}{DejaVu-Sans}
\newfontfamily{\AgdaTypewriterFont}{Frankenfont}
\renewcommand{\AgdaFontStyle}[1]{{\AgdaTypewriterFont{}#1}}
\renewcommand{\AgdaKeywordFontStyle}[1]{{\AgdaTypewriterFont{}#1}}
\renewcommand{\AgdaStringFontStyle}[1]{{\AgdaTypewriterFont{}#1}}
\renewcommand{\AgdaCommentFontStyle}[1]{{\AgdaTypewriterFont{}#1}}
\renewcommand{\AgdaBoundFontStyle}[1]{{\AgdaTypewriterFont{}#1}}

\AgdaNoSpaceAroundCode{}

% Adjust spacing after toc numbering
\usepackage{tocloft}
\setlength\cftchapnumwidth{3em}
\cftsetpnumwidth{4em}

% Style links with colors instead of boxes:
% https://tex.stackexchange.com/q/823
\definecolor{LINKCOLOR}{HTML}{2A7AE2}
\hypersetup{
  colorlinks,
  linkcolor={LINKCOLOR},
  citecolor={LINKCOLOR},
  urlcolor={LINKCOLOR}
}

\begin{document}

% Adjust indentation of Agda code blocks
\setlength{\mathindent}{0pt}
\setlength{\parindent}{0em}
\setlength{\parskip}{1em}

% Provide \tightlist environment
% https://tex.stackexchange.com/q/257418
\providecommand{\tightlist}{%
  \setlength{\itemsep}{0pt}\setlength{\parskip}{0pt}}

% Based on \titleAM:
% https://ctan.org/pkg/titlepages
\begin{titlepage}
  \newlength{\drop}%
  \setlength{\drop}{0.12\textheight}%
  \centering%
  \vspace*{\drop}
  \begingroup% Ancient Mariner, AW p. 151
  {\large Philip Wadler, Wen Kokke, and Jeremy G. Siek}\\[\baselineskip]
  {\Huge PROGRAMMING LANGUAGE}\\[\baselineskip]
  {\Huge FOUNDATIONS}\\[\baselineskip]
  {\Large IN}\\[\baselineskip]
  {\Huge Agda}\\[\drop]
  \vfill%
  {\small\scshape 2021}\par%
  \null\endgroup
\end{titlepage}

$for(parts)$

% Open front, main, and back matter sections:
$if(frontmatter)$
\frontmatter%
\setcounter{tocdepth}{0}

\tableofcontents%
\setcounter{tocdepth}{1}
$endif$
$if(mainmatter)$
\mainmatter%
$endif$
$if(backmatter)$
\appendix
\addcontentsline{toc}{part}{Appendices}
$endif$

% Only print title for main and back matter:
$if(frontmatter)$
% NOTE: Front matter titles are inserted via book/lua/single-file-headers.lua.
$else$
\part{$title$}
$endif$

% Include each section.
$for(sections)$
\hypertarget{$anchor$}{%
  \chapter{$title$}\label{$anchor$}}
\input{$tex_path$}
$endfor$

% Close front, main, and back matter sections:
$if(frontmatter)$
$endif$
$if(mainmatter)$
\cleardoublepage%
\phantomsection%
$endif$
$if(backmatter)$
$endif$
$endfor$

\end{document}
